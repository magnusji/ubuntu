\documentclass[norsk,12pt]{article}
\usepackage[utf8]{inputenc}
\usepackage[T1]{fontenc}

\usepackage{floatpag}				% Different pagestyles
\usepackage{amsmath}
\usepackage{array}	
\usepackage{booktabs}
\usepackage{amssymb}
\usepackage{graphicx}
\usepackage{epstopdf}
\usepackage{tabularx}
\usepackage{float}
\usepackage{caption}
\usepackage{subcaption}
\usepackage{parskip}
\usepackage{multirow}
\usepackage{listings}


\begin{document}


Hei
 
Jeg heter Magnus Isaksen og har vært medlem i LK Oslo i 2,5 år og søker herved på stilling som mottakskoordinator(MK) for IAESTE Norge.

Da jeg var nytt medlem begynte jeg som lokal PA. Deretter gikk jeg rett inn i MA vervet. Der lærte jeg mye om behandling av arbeidsgiver og tilreisende praktikanter. Det var flere utfordringer og mye moro. Utfordringene var mye knyttet til å skaffe bolig, men også i forhold til skattekontor, bank og utbetaling av lønn. Under min periode som MA satt også daværende MK i Oslo og jeg hadde flere diskusjoner med henne. Hun var veldig inspirerende og jeg fikk en oversikt over arbeidsmengden og hva som kreves.
 
Etter å ha vært borte i ett år på utveksling er jeg kommet tilbake og er motivert til å ta på meg ansvar i IAESTE nasjonalt. Dette er noe jeg har tenkt på lenge og allerede vurderte før jeg dro på utveksling. Da jeg kom tilbake var LK Oslo falt sammen og startet opp på nytt med mange nye medlemmer. Selv om jeg er der for å støtte disse i arbeidet for en ny bærekraftig LK, er jeg motivert til å ta steget videre og gjøre en innsats nasjonalt. Jeg har en viss innsikt i hva vervet krever av meg, da et tidligere medlem av IAESTE Oslo hadde vervet når jeg var MA. 




Her er forresten min favoritt ligning: 
$\nabla \times \vec{B} = 0$


IAESTE klem
Magnus Isaksen
\end{document}

