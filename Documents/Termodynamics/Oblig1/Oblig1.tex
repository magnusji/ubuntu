\documentclass[12pt]{article}
\usepackage[utf8]{inputenc}
\usepackage[T1]{fontenc}
\usepackage[english]{babel}
\usepackage{floatpag}				% Different pagestyles
\usepackage{amsmath}
\usepackage{array}	
\usepackage{booktabs}
\usepackage{amssymb}
\usepackage{graphicx}
\usepackage{epstopdf}
\usepackage{tabularx}
\usepackage{float}
\usepackage{caption}
\usepackage{subcaption}
\usepackage{parskip}
\usepackage{multirow}
\usepackage[]{mcode}

\begin{document}
\title{Oblig 1}

\subsubsection*{a)}
With N = 2 and q = 3 we get these possible microstates

\begin{table}[hb!]
\begin{tabular}{c|c}

N1 & N2 \\
\hline

3 & 0 \\
0 & 3 \\
2 & 1 \\
1 & 2 \\

\end{tabular}
\end{table}


\subsubsection*{b)}
With N = 3 and q = 3 we get these possible microstates

\begin{table}[hb!]
\begin{tabular}{c|c|c}
N1 & N2 & N3\\
\hline
3 & 0 & 0 \\
0 & 3 & 0 \\
0 & 0 & 3 \\
2 & 1 & 0 \\
2 & 0 & 1 \\
1 & 2 & 0 \\
0 & 2 & 1 \\
1 & 0 & 2 \\
0 & 1 & 2 \\
1 & 1 & 1 \\
\end{tabular}
\end{table}

\newpage

\subsection*{c)}
With N = 4 and q = 3 we get these possible microstates

\begin{table}[hb!]
\begin{tabular}{c|c|c|c}
N1 & N2 & N3 & N4 \\
\hline
3 & 0 & 0 & 0 \\
0 & 3 & 0 & 0 \\
0 & 0 & 3 & 0 \\
0 & 0 & 0 & 3 \\
2 & 1 & 0 & 0 \\
2 & 0 & 1 & 0 \\
2 & 0 & 0 & 1 \\
1 & 2 & 0 & 0 \\
0 & 2 & 1 & 0 \\
0 & 2 & 0 & 1 \\
1 & 0 & 2 & 0 \\
0 & 1 & 2 & 0 \\
0 & 0 & 2 & 1 \\
1 & 0 & 0 & 2 \\
0 & 1 & 0 & 2 \\
0 & 0 & 1 & 2 \\
1 & 1 & 1 & 0 \\
1 & 1 & 0 & 1 \\
1 & 0 & 1 & 1 \\
0 & 1 & 1 & 1 \\

\end{tabular}
\end{table}


\subsection*{d)}

$\Omega(N,q) = \frac{(q + N -1)!}{3!(N-1)!} $

Inserts N = 2 and q = 3 and gets $\Omega = 4$. Which in consistent with the listed microstates in a. 

When doing this for the values in b I find $\Omega = 10$ which seems correct. 

In c i find $\Omega = 20 $ which is also correct. 


\subsection*{e)}



\end{document}