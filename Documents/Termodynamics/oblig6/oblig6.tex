\documentclass[12pt]{article}
\usepackage[utf8]{inputenc}
\usepackage[T1]{fontenc}
\usepackage[norsk]{babel}
\usepackage{floatpag}				% Different pagestyles
\usepackage{amsmath}
\usepackage{array}	
\usepackage{booktabs}
\usepackage{amssymb}
\usepackage{graphicx}
\usepackage{epstopdf}
\usepackage{tabularx}
\usepackage{float}
\usepackage{caption}
\usepackage{subcaption}
\usepackage{parskip}
\usepackage{multirow}
%\usepackage[]{mcode}
\usepackage{listings}


\begin{document}
\title{Oblig2}
\author{Magnus Isaksen}
\maketitle
\section*{Exercise 0.8}

\subsection*{a)}

Vet at for ideell gass er det kjemiske potensialet uten noen annen påført kraft

$\mu = kT \ln(\frac{n}{n_Q})$ 

Hvor n er partikkel tettheten og $n_Q$ er en samling av konstanter. 

Setter vi på en elektrisk kraft $F = Nq\Phi$

Får vi $\mu = \frac{\partial F}{\partial N} = kT \ln(\frac{n}{n_Q}) + q\Phi$

Der q er ladningen og $\Phi$ er den elektiske kraften. 


\subsubsection*{b)}

For å finne likevekts tilstanden skal Helmholz fri energi vær lavest mulig.

Samtidig skal summen av krefter være 0. 

Dette finner jeg ved den deriverte av $\mu$ med hensyn på x. 

Slik at $\frac{d\mu}{dx} = 0$, da må $\mu$ være konstant.

\subsubsection*{c)}

Siden vi antar at n(x) er antisymmetrisk over $n(L/2) = n_0$. 

kan vi skrive 

$n_0 -n(x) = -(n_0 - n(L-x))$

$2n_0 = n(x) + n(L-x)$

og for $x= 0$ er 

\begin{equation}
2 n_0 = n(0) +n(L)
\label{eq1}
\end{equation} 

Og siden det skal være i likevekt er $\mu (0) = \mu (L)$

Dermed er

$kT\ln(\frac{n(0)}{n_Q} + q\Phi(0) = kT \ln(\frac{n(L)}{n_Q} + q\Phi(L)$ 

Hvor $\Phi(0) = 0$ og $\Phi(L) = \Phi_L$ 

Slik at 

$kT \ln(n(0)) = kT \ln(n(L)) +q\Phi_L$

Løser for $n(L)$

$\ln n(0) = \ln n(L) + \frac{q\Phi_L}{kT}$

$n(0) = n(L)e^{\frac{q\Phi_L}{kT}}$

Setter inn i \ref{eq1} 

$2n_0 = n(L)e^{\frac{q\Phi_L}{kT}} + n(L)$

som gir 

$n(L) = \frac{2n_0}{e^{\frac{q\Phi_L}{kT}} +1}$

Løser så for $n(0)$

$\ln n(L) = \ln n(0) - \frac{q\Phi_L}{kT}$

setter inn 

$2n_0 = n(0)e^{-\frac{q\Phi_L}{kT}} + n(0)$

$n(0) = \frac{2n_0}{e^{-\frac{q\Phi_L}{kT}} +1}$

\subsubsection*{d)}



\section*{Exercise 0.9}

\subsubsection*{a)}

Gibbs sum

$\sum_N \sum_{s(N)} e^{\frac{N\mu -\epsilon_{s(N)}}{kT}}$

gir 

$1 + e^{\frac{\mu - \epsilon}{kt}} + e^{\frac{2\mu - 4\epsilon}{kt}} + e^{\frac{3\mu - 9\epsilon}{kt}}$


\subsubsection*{b)}
Partikkel tallet 

$<N> = \sum_N NP(N) = 0*P(0) + 1*P(1)$

Sannsynligheten 
$P(1) = \frac{e^{\frac{0.2eV - 0.2eV}{kt}}}{1+e^{\frac{0.2eV -0.2eV}{kt}}} = \frac{1}{2}$

som da også gir

$<N> = \frac{1}{2}$

\subsection*{c)}

For kjemisk potensiale 0.6eV

$P(0.6) = \frac{e^{\frac{2*0.6 - 4*0.2eV}{kt}}}{1 +e^{0.2eV -0.2eV}{kt} + e^{\frac{2*0.6eV - 4*0.2eV}{kt}}} = \frac{1}{2}$

Dette gir ladnings tilstand

$<N> = 0*1 +1*\frac{1}{2} + 2*\frac{1}{2} = \frac{3}{2}$

\subsection*{d)}

For lave temperaturer går T mot 0 og dermed også gjennomsnittstilstanden, slik at grunntilstanden er eneste mulige. og dermed må alle være i denne tilstanden. 

\section*{Exercise 0.10}

\subsubsection*{a)}

$\sum_N \sum_{s(N)}e^{-\frac{\varepsilon_{s(N)}- N\mu}{kT}}$

Hvor N er partikkeltallet, $\varepsilon$ er bindings energien, $\mu$ er det kjemiske potensialet, k er Boltzmanns konstant og T er temperatur.

Grensene er satt slik at hvis N er 4 går grensen fra 0 til 3.  Fikk ikke et helt tydelig bilde av hvordan s(N) oppførte seg, men tror grensene her går også som for N.

\subsubsection*{b)}

For at

$ P(1) = \frac{e^{\frac{\mu - \varepsilon(1)}{kT}}}{1 + e^{\frac{\mu - \varepsilon(1)}{kT}}} = \frac{X}{Z_G}$

Hvor $X = e^{\frac{\mu - \varepsilon(1)}{kT}}$


\subsubsection*{c)}

For nøyaktig fire molekyler 


$P(4) = \frac{e^{\frac{4\mu - 16\varepsilon(1)}{kT}}}{1 + e^{\frac{\mu - \varepsilon(1)}{kT}} + e^{\frac{2\mu - 4\varepsilon(1)}{kT}} + e^{\frac{3\mu - 9\varepsilon(1)}{kT}} + e^{\frac{4\mu - 16\varepsilon(1)}{kT}}} = \frac{X^4}{Z_G} $ 

\subsubsection*{d)}

Herfra har jeg ikke fått gjordt resten. Var mye som var litt vanskelig å forstå i denne obligen. 

\end{document}