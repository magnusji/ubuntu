\documentclass[12pt]{article}
\usepackage[utf8]{inputenc}
\usepackage[T1]{fontenc}
\usepackage[norsk]{babel}
\usepackage{floatpag}				% Different pagestyles
\usepackage{amsmath}
\usepackage{array}	
\usepackage{booktabs}
\usepackage{amssymb}
\usepackage{graphicx}
\usepackage{epstopdf}
\usepackage{tabularx}
\usepackage{float}
\usepackage{caption}
\usepackage{subcaption}
\usepackage{parskip}
\usepackage{multirow}

\usepackage{listings}

\begin{document}

\title{Oblig 4}
\author{Magnus Isaksen}
\maketitle
\vfill
\newpage

\section*{a)}

Partisjonen finner vi ved 

$Z_R = \sum_{j=0}^{\infty} g(j)e^{\frac{-E}{kT}}$

hvor 

$E = j(j+1)\theta_rk$

$\theta_r k = \frac{\hbar^2}{2I}$

og 

$g = 2j+1$

Når vi setter disse inn blir uttrykket

$Z_R = \sum_{j=0}^{\infty} (2j+1)e^{\frac{-j(j+1)\frac{\hbar^2}{2I}}{kT}}$

\section*{b)}

Skriver python programmet 

\begin{lstlisting}
from pylab import *
from numpy import *

Tmarked = [0.05, 0.2, 5, 10 , 15, 20]
n = 100
lT = len(Tmarked)-1
z = zeros((lT,n), 'float')
for i in range(lT):
    for j in range(n):
        z[i,j] = (2.0*j+1)*exp(-(j*(j+1))*Tmarked[i])
        print j



t = linspace(0,5,n)

hold('on')
for i in range(lT):
    plot(t,z[i,:])

show()

\end{lstlisting}

\section*{c)}

Når T blir mye større enn $\theta_R$ vil det som er i eksponenten i uttrykket gå mot null, altså 

$\lim_{T/\theta \rightarrow 0} Z_R = \int_0^\infty (2j+1)$



\end{document}